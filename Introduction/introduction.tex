\chapter{Introduction}

\section{Context}
This research work is part of the RADIAL (RADiomics Image Analysis of the Liver) project, carried out by the
institute for minimally invasive surgery from Strasbourg (IHU), where
one of the objectives is to improve the treatment and the prevention of
liver cancer.\\
Being the second leading cause of cancer-related death worldwide, liver
cancer is considered as a major public health challenge.\\
Current ways to characterize the liver cancer are either invasive, 
through the extraction of tissues samples or non-invasive, typically 
through interpretation of medical images.\\
%radiomics
Radiomics is a new field developed in the early 2010s to extract the encoded information within medical images in order to provide a more objective and image-guided assessment of the disease.\\
Conventional radiomics however suffers from a lot of limitations such as the hand-crafted design of the extracted features or the computation of characteristics only in a manually defined region or volume of interest. Deep learning changed the way to apprehend several medical imaging related problematics thanks to its ability to detect
morphological properties in images only by using the pixel intensities
as input. The radiomics field was impacted by this new set of algorithms, and a new branch of radiomics was created, where the definition of the region of interest and/or the extraction of features are performed using deep learning approaches.

\section{Contributions}
In our work we present strategies to automatically provide characterization of the liver tumors.
We detailed our review presenting current liver-related hand-crafted radiomics studies and proposed different ways to improve their reproducibility.
Often performed manually, the delineations of the tumors can suffer from various bias, such as the inter and intra-observer variability, or its undefined borders leading to errors when diagnosing, classifying or staging the liver cancer. Therefore, we present a robust automatic pipeline to perform semantic segmentation of liver tumors using both the power of deep cascaded networks architecture and the incorporation of dynamic contrast-enhanced information. We proved the value of our cascaded architecture by providing missing annotations in a weakly-annotated multiphase CT datasets using our approach.
To characterize the tumors, we tackled the automatic histological grade prediction and we were the first to propose a deep radiomics pipeline which used relevant semantic imaging features to assess the grade in a slice-wise fashion.

\section{Outline of the thesis}
We present the different types of liver cancers, and particularly the
hepatocellular carcinoma (\ac{hcc}), which will be the main topic of our work.
We describe the different steps of hepatocarcinogenesis, which
transforms healthy hepatic tissues into cancerous ones, by analyzing the
evolution of the cells and the associated pathological changes. 
In case of suspected cancer, the current ways to establish a diagnosis 
are either through extraction and inspection of tissue samples or 
non-invasively by analyzing medical images.  We outline the different modalities usually exploited to establish a
diagnosis, and focus on the computed tomography (\ac{ct}) for the rest of this
research work.  We first present the different imaging modalities available to perform a
diagnosis, before detailing the two branches of the radiomics field. We
review liver radiomics studies and focus on the deep learning radiomics
branch for its ability to extract features directly from the input without 
necessary requiring any prior knowledge. We present our multiphase 
cascaded architecture to segment the liver and
its tumor, before applying this to weakly annotated datasets. We finally
use extracted features from our multiphase architecture to predict the
histological grade. \\
The analysis of medical images is commonly performed by the naked eye,
but technological advances allow the creation of computer assisted
diagnosis tools to investigate the link between imaging features 
and biological characteristics.
We present the radiomics technique allowing the extraction of
quantitative features from images. The conventional pipeline is based on 
engineered features, thus considering the conventional radiomics pipeline as
hand-crafted, hence the name of \emph{\ac{hcr}} (Hand-Crafted Radiomics).
To depict the lack of reproducibility present in the vast majority of
the \emph{\ac{hcr}}-liver related studies, we expose the conclusions of our
review where \emph{\ac{rqs}} (Radiomics Quality Score) were analyzed. \\
The emergence of deep learning (\ac{dl}) allows creation of algorithms where
features are learned directly from the raw images without extra information. 
The radiomics field benefited from the development of DL, and a new branch 
was created, the \emph{\ac{dlr}} (Deep-Learning Radiomics).
We review the DLR studies performed on the \ac{hcc}, and raise the issue of
segmentation, often manually performed and thus suffering from high
inter and intra-variability.
To combat this variability a lot of work has been done to perform
automatic segmentation of liver and its tumors.
The first methods developed to perform this task were based on
anatomical prior-knowledge or manual interactions, and often failed due to 
the limited amount of training samples, or when
facing pathological cases.
Automatic deep learning segmentation can address most of these issues by
looking for relevant features directly from the images only.
We review the different \ac{dl} related liver tumors semantic segmentation
studies and extract the common key settings to build our own segmentation pipeline.
From these key settings, we focus on the cascaded architecture allowing,
in the case of tumors segmentation, the implementation of two specialized
networks (one to segment the liver on an entire \ac{ct} slice and the second to delineate the
tumors, but only on the liver region).\\
To prove the ability of \ac{dl} networks to segment liver tissues, and to
prove that a cascaded architecture combined with multiphase CT images
allows the best accuracy, we perform semantic segmentation of liver
tissues on an internal dataset.\\
To complete missing annotations in a weakly annotated or non-annotated
dataset, we use the same architecture.
And finally, we focus on the prediction of the histological grade, being
an indicator of the evolution of the disease. We investigate whether the
features extracted from our cascaded multiphase semantic segmentation
are sufficiently relevant to perform this task.
