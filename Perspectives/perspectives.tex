\chapter{Perspectives}

Several axes of improvement can be imagined regarding our research work.

When performing a hand-crafted or a deep radiomics study, we still
believe that the key part will be the area selection preceding the
computation of the radiomics features.

\section{Semantic Segmentation}

As discussed before, the HCR paradigm requires manual experts
segmentations, and several studies already discussed the inter- and
intra-observer variability present among the obtained delineations. This
variability is translated in the quality of the retained features, that
tend not to be robust enough when performing the final predictions.

The DLR paradigm allows the segmentation to be made automatically, thus
reducing the number of potential biases. The automatic segmentation, as
performed in our research work, however requires a high number of
training cases to offer a realistic delineation of the targeted tissues.

In the case of liver tumors, a cascaded architecture stacking a first
network responsible for the segmentation of the liver and a second
dedicated only to detect the tumor within the obtained liver mask is
currently the method allowing to obtain the most accurate results.
State-of-the-art liver segmentation methods currently allow us to reach
annotations similar to those obtained by the experts with mean DSC often
above 0.95-0.96. We believe that there is no real need to improve the
liver segmentation in the future, and that more interest needs to be
shown in tumor and other liver tissues segmentation.

Regarding the segmentation of the tumors, current state-of-the-art
results were obtained thanks to post-processing steps such as false
positives (FP) filtering post-processing method to reduce the number of
object misclassified as lesions.

Therefore current semantic segmentation architectures still lack the
ability to discriminate between real lesions and other areas sharing the
same textural properties such as regions close to the vessels as
illustrated in the figure \ref{fig:image5}.



\begin{figure}[th!]
\centering
\includegraphics[width=0.5\linewidth]{Perspectives/images/image5}
\caption{Tumor candidates marked with a dashed white line were classified as false positives, as described by Chelbus et al. \cite{Chlebus2018}.}
\label{fig:image5}
\end{figure}

One reason explaining this low recall might be the poor quality of the
currently available segmentation datasets. Liver annotations often
contain non-hepatic areas (such as air most of the time) or
imperfections close to the organ borders.

When performing the automatic liver segmentation, the deep neural
networks will most of the time be able to avoid these regions. However,
in case they are integrated in the liver mask, they will often be
misclassified as tumors by the second network in the cascade
(responsible for the tumor segmentation).

To overcome this issue, the publicly available datasets such as the
\lmttfont{LIST-dB} need to be properly resegmented. They usually only contain
annotations for the parenchyma or the tumors, but a third class (or
more) could be considered to incorporate tissues belonging to none of
these groups such as the cysts or the blood vessels.\\
As explained in the section  \ref{subsection:StateOfTheArtDlImplementations}, current state-of-the-art tumor
segmentation results are often obtained with 2D or 2.5D networks, but
the post-processings steps usually help to improve the obtained accuracy
(3D CRF or even simple morphological operations). Several studies
implemented a U-Net like architecture, where the information is
compressed before being decoding to obtain the final segmentation map.
One of the key concepts in these architectures is the reinjection of
features learned in the earlier layers later in the network, either
thanks to skip-connections, residual units or even densely connected
layers. Another way to perform the segmentation could be to implement a
multi-scale pyramidal architecture where tumor-related features can be
learned directly from the raw images at multiple scales instead of being
learned from previously obtained features, which is often the case in
the aforementioned architectures.

It might also be interesting to perform the segmentation with a full 3D
architecture. Some studies already investigated this solution \cite{Dou2016}, but they could not yet overtake state-of-the-art
results. 

Some other concepts could be incorporated in new architectures. Jin et
al. \cite{Jin2018} for example proposed a network that integrated both U-Net and
attention residual mechanism to proceed the segmentation of both the
liver and the lesions. The residual attention mechanism has been
introduced in 2017 by Wang et al. \cite{Wang2017} to perform image
classification, with the idea that the attention mechanism can help the
network focusing on specific parts of the image. The study from Jin et
al. was the first to use the attention mechanism for
semantic segmentation purposes. On the hidden test set of LITS, they
outperformed a lot of 2D-based methods, but were still far from the
top-ranked teams. However, new paradigms such as the self-attention
mechanism, in combination with state-of-the-art 2D and 3D architectures
are certainly an avenue for the improvement of the automatic liver and
tumors segmentation tasks \cite{Chen2019}.

\section{Dynamic contrast-enhanced images}

In our research work, the key element is the extraction of the features
from dynamic contrast-enhanced images. However, it has been very
difficult to collect dynamic CECT based images, since only a few
publicly available datasets contain this type of volume.

There is a huge room of improvement regarding this specific type of
dataset. It has been mentioned in the Medical Context Section, that the
use of images obtained after the injection of contrast medium largely
improves the visual and the automatic quality of the diagnosis.
Nonetheless, we realized that multiphasic images could be better used in
either the semantic segmentation and/or the radiomics field.

The first issue is related to the acquisition protocol required to
obtain this type of images. Most of the time, the different retained
phases, namely arterial, portal venous or delayed phases are acquired
following either visual inspection of the radiologists \footnote{The arterial phase can sometimes be acquired
with the help of a bolus tracking technique}, or after a specific
duration following the injection moment that can differ from one study
to the other. Another criteria to consider is the physiological
characteristics of the patient such as its weight that will determine
how the contrast medium is diffused through the liver.

When dealing with multiphase databases, we believe that these key
elements need to be considered, at least the moment of injection and/or
the duration between the injection and the acquisition since the weight
of the patient can be inferred from the liver volume (which can be
predicted thanks to semantic segmentation). Currently, only the DICOM
format allows the addition of such metadata (only the exact acquisition
moment but not the injection moment) but they are barely reported and
are often left to default values.

Obtaining such data seems really challenging, hence, some techniques
need to be implemented to deal with current multiphase images.

For example, we believe that a ``phase correction system'' could be
designed that would potentially normalize images belonging to the same
phase. The automatic detection of the amount of contrast medium still in
the liver could also help to predict the exact duration between the
injection of the CM and the acquisition.

In our research work, we have decided to implement an histogram
specification algorithm to ``normalize'' images from a given phase. The
first step requires to compute a mean histogram per phase (requires to
have a sufficient amount of patients images correctly labeled), before
transforming the volumes of a given patient to fit the obtained mean
histogram, as depicted in the figure \ref{fig:specif}.

\begin{figure}
\centering
\begin{minipage}{0.4\linewidth}
\includegraphics[width=\linewidth]{./images/image3.png}
\end{minipage}
\begin{minipage}{0.4\linewidth}
\includegraphics[width=\linewidth]{./images/image1.png}
\end{minipage}
\begin{minipage}{0.4\linewidth}
\includegraphics[width=\linewidth]{./images/image2_modified.png}
\end{minipage}
\begin{minipage}{0.4\linewidth}
\includegraphics[width=\linewidth]{./images/image4.png}
\end{minipage}
\caption{Histogram specification experiment conducted to normalize images from the same phase, top row: slice before histogram specification; second row: mapping function for the chosen range [-100, 400]; third row: representation of the mean phase histogram, the current slice histogram and the transformed one after application of histogram specification; bottom row: slice after histogram specification.}
\label{fig:specif}
\end{figure}

In the case of multiphase images, the expert performing the segmentation
will usually try to obtain only one segmentation for the registered
volumes. Being coherent for the liver, we believe that this technique
needs to be modified for internal liver tissues. The lesions for example
will have a different behavior from one phase to the other (in case for
example of hypo- or hyper-dense tumors), several segmentation should
therefore be performed. This might potentially help the deep neural
networks to better understand the dynamic of the lesions, and can also
address the problem of small mis-registration when only one ground truth
segmentation map is shared by all the available phases.

One long-term objective could be to create, with the help of
radiological experts, a dataset containing registered multiphase CT
volumes with expert annotations performed following the aforementioned
protocol.

\section{Deep Radiomics}

Regarding our DLR study, we believe that the prediction of the
histological grade can be improved.

One lead of improvement could be the prediction of the grade at a finer
scale, but this will require to know the exact position of the extracted
sample, or to obtain a map of the heterogeneous regions after surgical
removal of liver tissues through liver resection or liver transplant for
example.

In our research work we investigated a patch-wise prediction of the
histological grade but the results were not as promising as those
obtained through our slice-wise approach, especially because our
patch-wise semantic segmentation network needs to be improved.

We are currently considering the relevant imaging features as being the
ones extracted from the bottleneck part of our U-Net network but other
techniques such as the auto-encoder could be investigated to better
extract the relevant information from the raw images, or from the
retained features. We could have also potentially incorporated clinical
data (AFP levels, age, \ldots{}) in our pipeline, however such data are
often difficult to retrieve, and difficult to combine with imaging
features, consequently, we decided to focus only on images in the
current work.

If more interest is shown in the future towards the prediction
histological grade, we believe that the existing grading systems have to
be standardized by considering more criteria than just the worst or the
most frequent grade present in the histological slices, which is
currently the case.

Our DLR architecture obtained promising results for the prediction of
the histological grade, but we believe that the same architecture can be
adapted to focus on other prediction tasks, such as the prediction of
recurrence after treatment using longitudinal studies.
