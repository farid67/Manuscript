\renewcommand{\arraystretch}{2}\setlength{\tabcolsep}{20pt}
\setlength{\tabcolsep}{8pt}
\newgeometry{top=5mm, left=5mm, right=5mm, bottom=5mm, footskip=0mm, headsep=0mm}
\begin{landscape}
\scriptsize
\begin{longtable}{p{1.2cm}|p{1.3cm}@{\hspace{2em}}p{1cm}p{1.6cm}p{1.7cm}p{3cm}p{2cm}p{2cm}p{1.4cm}p{1.5cm}p{2.5cm}p{1.5cm}}
\textbf{Author} &\textbf{Modality \newline \& slice \newline thickness} &\textbf{Mean \newline tumor\newline size} &\textbf{Treatment} &\textbf{\#Patients \newline \& Inclusion Criteria} &\textbf{Segmentation} &\textbf{Computed features} &\textbf{Retained \newline features} &\textbf{Retained \newline features \newline category} &\textbf{Study endpoints} &\textbf{Results} &\textbf{\%RQS \newline (total points)} \\ \hline \endhead


Cozzi et al. \cite{Cozzi2017} & NECT \newline 3mm &- &Radiotherapy (volumetric modulated arc therapy) &\textbf{138} Patients \newline BCLC stages from A to C, Child-Pugh stages A-B &Segmentation done using the CTV (clinical target volume) which is manually contoured (whole tumor analysis) &\textbf{35} extracted features \newline 6 geometry and histogram\newline 29 GLM &Compacity \newline Energy \newline GLNU &Quantitative &OS \& local control of the tumor& AUC of the model is 0.80 \newline Survival could be predicted with a radiomics signature &14 (5) \\ 


Zhou et al. \cite{Zhou2017a} &Contrast CT\newline(30 and 60s) 1.25mm &- &Hepatectomy &\textbf{215} Patients who underwent partial hepatectomy &Largest cross-sectional area of the tumor, manual delineation \newline Exclusion of necrosis \newline 2 experts &\textbf{300} features (Mean, SD, Kurtosis, Skewness, GLM) &Histogram features (skewness, energy, means, \ldots) &Quantitative &Recurrence & First-order statistical features combined with clinical factors can predict early recurrence &25 (9) \\


Akai et al. \cite{Akai2018} &Contrast CT \newline (27-28, 40 and 90s) \newline 5mm &3.7cm \newline (2.4-7.0cm) &Hepatectomy &\textbf{127} patients &Manually setting the ROI to include the tumor within the slice at its max diameter. \newline Single radiologist &\textbf{96} features (mean, sd, positive calue pixels, entropy, kurtosis, skewness) &Entropy, skewness and kurtosis &Quantitative &OS \& DFS &First-order statistical features were sufficient to predict postoperative survival &25 (9) \\
Chen et al. \cite{Chen2017} &Contrast CT (25 and 70s) 1.25mm &- &Hepatectomy &\textbf{61} patients with only one lesion and survival above 3 months &ROI was delineated around the tumor outline at the longest dimension \newline 2 experts &\textbf{84} features \newline 12 Gabor \newline 9 Wavelet \newline 7 GLCM &Textural features, Gabor and Wavelet as key features &Quantitative &OS \& DFS &Tumor prognosis could be predicted using Gabor and Wavelet responses &17 (6) \\


Li et al. \cite{Li2016}&Contrast CT (70s)\newline 1.25mm &8.0cm \newline (5.1-18.7cm) &Hepatectomy or TACE &\textbf{130} patients \newline86 treated by LR and 22 by TACE &Irregular ROI manually drawn around the largest-cross sectional tumor outline \newline 2 radiologists &\textbf{27} features (Wavelet) &2 Wavelet features correlated with survival &Quantitative &OS and Treatment sensitivity &Wavelet features correlated with survival suggesting a suitable treatment choice &19 (7) \\


Raman et al. \cite{Raman2015} &Contrast CT (25s) 3mm &Adenoma $7\pm3$cm \newline FNH $6\pm3$cm \newline HCC $8\pm3$cm &- &\textbf{80} patients \newline 17 FNH \newline 19 Adenomas \newline 25 HCCs \newline 19 normal livers &ROIs were selected from multiple axial slices (from 5 to 10 slices) \newline 2 experts &\textbf{32} features (mean, SD, entropy, skewness, kurtosis) & SD and Mean of histogram &Quantitative &Diagnosis &First-order statistical features able to differentiate 3 types of hypervascular lesions with a 15\% of error rate) &3 (1) \\


Kuo et al. \cite{Kuo2007} &Contrast CT (30-35 and 60-70s) 2.5mm &- &- &\textbf{30} Patients \newline no patients received chemo before resection &No segmentation \newline Images analyzed visually by 2 experts. &\textbf{6} imaging traits (internal arteries, textural heterogeneity, wash-in-wash-out, necrosis, tumor margin) &Tumor margin \newline Internal arteries &Semantic &MVI status &The tumor margin showed strong correlation with MVI, TNM. \newline Internal arteries showed correlation with MVI &19 (7) \\


Banerjee et al.  \cite{Banerjee2015} &Contrast CT (30-35, 60-70, 180-300s) \newline 2.5-3mm &2.8cm (1.8-4.5cm) &Hepatectomy or LT\textsuperscript{1} &\textbf{157} patients \newline 72 resection\newline 85 LT\footnote{Liver transplantation} \newline MVI diagnosed in 45 patients &Only imaging features were evaluated by 5 radiologists &\textbf{3} imaging traits (internal arteries, hypodense halo, tumor-liver difference)& The 3 imaging traits were retained &Semantic &OS and RFS &Combination of the three different imaging traits was correlated with MVI &53 (19) \\


Renzulli et al. \cite{Renzulli2016} &Contrast CT (25-30, 45-60, 180-300s) 2.5mm &3.3cm (1.8-5.2cm) &Hepatectomy &\textbf{125} patients where hepatic resection was indicated &Only imaging features were evaluated by 2 radiologists &\textbf{5} imaging traits \newline Dimensions \newline Lesions number \newline Non-smooth margins \newline TTPVI \footnote{Two-Trait Predictor of Venous Invasion: Internal arteries and Hypoattenuating halo} \newline Peritumoral enhancement& All except the lesion dimensions &Semantic &MVI status &The 4 retained traits were correlated with the presence of MVI in HCC &8 (3) \\



Segal et al.  \cite{Segal2007} &Contrast CT (threephasic) &- &Hepatectomy &Train: \textbf{30} \newline Test: \textbf{32} &No segmentation, visual examination \newline 2 radiologists. &\textbf{32} imaging traits (Capsule, Wash-in-Wash-out, Tumor-Liver difference, \ldots) &Internal arteries and hypodense halo &Semantic &OS \& MVI &Internal arteries combined with  hypodense halo can predict OS, MVI \newline Both are correlated with HCC gene expression&42 (15) \\



Zheng et al.  \cite{Zheng2018} &Contrast CT (22 and 60s) 5mm &- &Hepatectomy &Train: \textbf{212}\newline Test: \textbf{107} \newline patients without anticancer therapy &ROI delineated around the tumor outline of the largest cross-sectinal area. \newline 2 radiologists &\textbf{110} GLM features &6 GLCM features &Quantitative &Recurrence \& OS &Textural features sufficient to predict postoperative recurrence and survival &47 (17) \\


Peng et al.  \cite{Peng2018} &Contrast CT (30, 60 and 120s) \newline 5mm &4.9-6.4cm &- &Train: \textbf{184} \newline Test: \textbf{120} \newline Partial hepatectomy with pathologically confirmed HCC &ROI semi-automatically segmented in the largest cross-sectional area \newline 2 experts&\textbf{5} imaging traits (tumor margin, peritumoral enhancement, hypoattenuating halo, internal arteries, tumor-liver difference) \& \textbf{980} quantitative features &Nonsmooth tumor margins, hypoattenuating halos and internal arteries + 8 radiomics features (Entropy, shape, GLRLM, GLCM) &Semantic \& \newline Quantitative &MVI status &Radiological features and a radiomics signature computed with first-order statistical features showed correlation with MVI &47 (17) \\



Bakr et al.  \cite{Bakr2017} &Contrast CT (AR with bolus tracking, PV, delay)\newline Thickness $ \leq 3\text{mm}$&7.4cm &- &\textbf{28} patients with surgical resection of a previously untreated HCC &3 ROIs were placed on different cross sections of the tumors (one central, one superior and one inferior)\newline 4 radiologists &\textbf{464} features (intensity, texture, shape) &Textural features &Quantitative &MVI status &Textural features computed using single- or combined-phased images were correlated with MVI &3 (1) \\



Taouli et al.  \cite{Taouli2017} &Contrast CT (AR with bolus tracking, PV at 70s, Delay at 180s) &5.7$\pm$3.2cm &- &\textbf{38} patients \newline \textbf{26} CT/ 12 MRI \newline 36 Liver resection \newline 2 LT\footnote{Liver Transplantation} & Global inspection of the imaging traits and “slice-wise” evaluation for the enhancement ratio and the wash-out ratio \newline 2 radiologists &\textbf{11} imaging traits (wash-in-washout, hypovascularity) \newline \textbf{3} Quantitative features (contrast ratios)&Infiltrative pattern, mosaic appearance, presence of MVI, large size &Semantic \& \newline Quantitative &Signature of MVI \& aggressive phenotype &Correlation was found between some imaging traits and the aggressive profile of the tumors &19 (7) \\



Xia et al. \cite{Xia2018} &Contrast CT (30, 55~70, 300s)\newline 2.5-5mm &12 tumors smaller than 5cm, 26 larger &Hepatectomy or LT &\textbf{38} patients &Tumor was firstly delineated then divided into 3 spatially distinct sub-regions (using a multi-parametric clustering) \newline 1 radiologist &\textbf{37} features (1st order, geometry, textural) And \textbf{4} features for the whole tumor &Volume of transition region \& cluster prominence &Quantitative &OS &The volume of transition between tumor and liver, and the heterogeneity of the lesion were correlated with survival. &22 (8) \\
\caption{HCR reviewed studies details}\label{tab:HCR_studies_details}
\end{longtable}

\end{landscape}
\renewcommand{\arraystretch}{5}
\newgeometry{vmargin={15mm}, hmargin={30mm,30mm}}   % set the margins 
