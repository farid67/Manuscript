\renewcommand{\arraystretch}{1}
\newgeometry{vmargin={15mm}, hmargin={10mm,10mm}}
\begin{landscape}
   % set the margins 
\small

\begin{longtable}{p{2cm}|p{1.5cm}p{1cm}p{2cm}p{2cm}p{2cm}p{2cm}p{2cm}p{2cm}p{1.5cm}p{2cm}p{1cm}}
\textbf{Author} &\parbox{2cm}{\textbf{Modality \\ \& slice \\thickness}} &\parbox{2cm}{\textbf{Mean \\tumor\\ size}} &\textbf{Treatment} &\textbf{\#Patients + Inclusion Criteria} &\textbf{Segmentation} &\textbf{\#Computed features} &\textbf{Retained features} &\textbf{Retained features category} &\textbf{Study endpoints} &\textbf{Results} &\textbf{\% RQS (points)} \\ \hline \endfirsthead
Cozzi et al. \cite{Cozzi2017} &NECT 3 mm &- &Radiotherapy (volumetric modulated arc therapy) &138 Patients with BCLC stages from A to C, Child-Pugh stages A-B &Segmentation done using the CTV (clinical target volumes) which is manually contoured (whole tumor analysis) &35 extracted features * 6 geometry and histogram based features * 6 GLCM * 2 NGLDM * 11 GLRLM * 10 GLZLM &Compacity (shape-based feature) , Energy (histo-based) and GLNU &Quantitative &OS \& local control of the tumor after radiation treatment &* AUC of the model is 0.80 * Survival could be predicted using a radiomics signature made by a single shape-based feature. &14 (5) \\ 
Zhou et al. \cite{Zhou2017a} &Contrast CT (30 and 60s) 1.25 mm &- &Hepatectomy &215 Patients who underwent partial hepatectomy &Largest cross-sectional area of the tumor, manual delineation - exclusion of necrosis 2 experts &300 features * Mean, SD, Kurtosis, Skewness * Percent Mean and SD * 5 features from 4 GLM &Radscore uses histogram features (skewness, energy, means...) &Quantitative &Recurrence &* Radiomics signature using first-order statistical features combined with clinical factors was a good predictor of early recurrence after surgery &25 (9) \\
Akai et al. \cite{Akai2018} &Contrast CT (27-28, 40 and 90 s) 5 mm &3.7 cm (2.4 - 7.0 cm) &Hepatectomy &127 patients &Manually setting the region of interest to include the tumor within the slice at its max diameter. Single radiologist &96 features (mean, sd, positive calue pixels, entropy, kurtosis, skewness) &Entropy and histogram-based features (skewness and kurtosis) &Quantitative &OS and DFS &First-order statistical features were sufficient to predict postoperative survival &25 (9) \\
Chen et al.  23 &Contrast CT (25 and 70s) 1.25 mm &- &Hepatectomy &61 patients with only one lesion and prospective survival above 3 months &ROI was delineated around the tumor outline at the longest dimension (2 experts) &84 features * 12 Gabor * 9 Wavelet * 7 GLCM &* Textural features * Gabor and Wavelet as key features &Quantitative &OS and DFS &Tumor prognosis could be predicted using Gabor and Wavelet responses &17 (6) \\
Li et al.  24 &Contrast CT (70s) 1.25 mm &8.0 cm (5.1 - 18.7 cm) &Hepatectomy or TACE &130 patients 86 treated by LR and 22 by TACE &A user-defined irregular ROI was drawn manually around the largest-cross sectional tumor outline by each expert 2 radiologists &27 features (Wavelet) &2 Wavelet features correlated with survival &Quantitative &OS and Treatment sensitivity &Wavelet features were correlated with post operative survival suggesting a suitable treatment choice &19 (7) \\
Raman et al.  26 &Contrast CT (25s) 3 mm &* Adenoma 7 $ \pm $ 3cm * FNH 6 $ \pm $ 3 cm * HCC 8 $ \pm $ 3cm &- &80 patients 17 FNH 19 Adenomas 25 HCCs 19 normal livers &ROIs were selected from multiple axial slices (from 5 to 10 slices) 2 observers &32 features (mean, SD, entropy, skewness, kurtosis) & SD and Mean of histogram &Quantitative &Diagnosis &A model created using exclusively first-order statistical features was able to differenciate 3 types of hypervascular lesions. (They reached 15\% of error rate) &3 (1) \\
Kuo et al.  29 &Contrast CT (30-35 and 60-70s) 2.5 mm &- &- &30 Patients * no patients included received chemo before resection &No segmentation, Images analyzed visually by 2 experts. &6 imaging traits * Internal Arteries * Textural heterogeneity * Wash-in - Wash-out * Necrosis * Tumor margin score &Tumor margin showed a strong correlation with venous invasion and TNM stage internal arteries showed correlation with venous invasion &Semantic &MVI status &The tumor margin showed strong correlation with MVI, TNM, and the expression of a drug response gene. While, internal arteries showed correlation with MVI &19 (7) \\
Banerjee et al.  33 &Contrast CT (30-35, 60-70, 180-300 s) Thickness of 2.5-3 mm &2.8 cm (1.8-4.5 cm) &Hepatectomy or LT &157 patients * 72 surgical resection * 85 liver transplantation MVI diagnosed in 45 patients &Only imaging features were evaluated by 5 radiologists &3 imaging traits * Internal arteries * Hypodense halo * Tumor-Liver difference &Internal arteries, hypodense halo, tumor liver diffeence &Semantic &OS and RFS &RVI (radiogenomic venous invasion) computed with three different imaging traits was correlated with MVI &53 (19) \\
Renzulli et al.  34 &Contrast CT (25-30, 45-60, 180-300s) 2.5 mm &3.3 cm (1.8-5.2 cm) &Hepatectomy &125 patients where hepatic resection was indicated &Only imaging features were evaluated by 2 radiologists &5 imaging traits * Dimensions * Number of lesions * Non-smooth margins * TTPVI (internal arteries + hypoattenuating halo) * Peritumoral enhancement &tumor dimension, non-smooth margins, peritumoral enhancement and TTPVI &Semantic &MVI status &Tumor size, non-smooth tumor argins, peritumoral enhancement and TTPVI were correlated with the presence of MVI in HCC &8 (3) \\
Segal et al.  38 &Contrast CT (threephasic) &- &Hepatectomy &\textbf{Training: 30 Test: 32} &Only imaging traits (a total of 138) extracted by two radiologists. &32 imaging traits * Capsule * Wash-in, Wash-out, * Tumor-Liver difference &Internal arteries and hypodense halo &Semantic &OS and MVI &Internal arteries was found as being a key imaging feature for predicting OS and MVI in combination with hypodense halo. Those features were also correlated with the expression of genes involved in the development of HCC lesions &42 (15) \\
Zheng et al.  40 &Contrast CT (22 and 60s) 5mm &- &Hepatectomy &Training: 212 Test: 107 patients without other malignancies nor anticancer therapy &ROI delinated around the tumor outline of the largest cross-sectinal area 2 radiologists &110 GLCM related features &6 GLCM features &Quantitative &Recurrence and OS &Radiomics score computed with textural features was sufficient to predict postoperative recurrence and survival in patients with solitary HCC &47 (17) \\
Peng et al.  41 &Contrast CT (30, 60 and 120s) 5mm &Training: *MVI(+): 6.3 cm *MVI(-): 5.7 cm Validation: *MVI(+): 6.4 cm *MVI(-): 4.9 cm &- &\textbf{Training: 184 Test: 120 Partial hepatectomy with tumor tissues pathologically confirmed to be HCC } &Images reviewed by 2 experts ROI semi-automatically segmented in the largest cross-sectional area &5 imaging traits * tumor margin * peritumoral enhancement * hypoattenuating halo * internal arteries * tumor-liver difference (binary) 980 quantitative features &nonsmooth tumor margins, hypoattenuating halos and internal arteries + 8 radiomics features (Entropy, shape, GLRLM, GLCM) &Semantic \& Quantitative &MVI status &Radiological features and a radiomics signature computed with first-order statistical features showed correlation with MVI &47 (17) \\
Bakr et al.  42 &Contrast CT (AR, PV, delay) optimal arterial opacification obtained using an automated bolus tracking technique Thickness $ \leq $ 3mm &7.4 cm &- &28 patients who underwent surgical resection of a previously untreated HCC &3 ROIs were placed on different cross sections of the tumors (one central, one superior and one inferior) 4 radiologists &464 features (intensity, texture, shape) &Textural features &Quantitative &MVI status &Textural features computed using single- or combined- phased images were correlated with MVI &3 (1) \\
Taouli et al.  45 &Contrast CT (AR with bolus tracking, PV at 70s, Delay at 180s) &5.7 $ \pm $ 3.2 cm &- &38 patients --> 26 CT/ 12 MRI * Liver resection (n = 36) * Liver transplantation (n = 2) &Imaging traits + “slice-wise” evaluation for the enhancement ratio and the wash-out ratio. 2 radiologists &11 imaging traits: * washin washout, hypovascularity... and 3 Quantitative features * enhancement ratios * Washout ratios * Tumor-to-liver contrast ratio &infiltrative pattern, mosaic appearance, presence of MVI, large size &Semantic, Quantitative &signature of MVI and/or aggressive phenotype &Correlation was found between some imaging traits and the aggressive profile of the tumors &19 (7) \\
Xia et al.  46 &Contrast CT (30, 55~70, 300 s) Thickness of 2.5~5mm &12 tumor below 5cm 26 above 5cm &Hepatectomy or LT &38 patients &Tumor was firstly delineated then divided into 3 spatially distinct sub-regions (using a multi-parametric clustering), whole tumor analysis. 1 radiologist &37 features (1st order, geometry, textural) And 4 features for the whole tumor &volume of transition region and cluster prominence &Quantitative &OS &The volume of transition between tumor and liver, and the heterogeneity of the lesion were correlated with survival. &22 (8) \\
\caption{HCR reviewed studies details}\label{tab:HCR_studies_details}
\end{longtable}

\end{landscape}
\renewcommand{\arraystretch}{5}
\newgeometry{vmargin={15mm}, hmargin={30mm,30mm}}   % set the margins 
