\renewcommand{\arraystretch}{1}
\begin{landscape}
\begin{table}[!htp]\centering

\caption{HCR reviewed studies details}\label{tab:HCR_studies_details}
\tiny
%column

%Author 
%Modality & slice thickness
%Mean tumor size	
%Treatment
%Target
%#Patients + Inclusion Criteria
%Segmentation
%Filtration
%number of computed features	
%Retained features
%Retained features category	
%Study endpoints	
%Results	
%% RQS (points)
\begin{tabular}{lp{1.5cm}p{1.5cm}p{2cm}p{2.5cm}p{3cm}p{3cm}p{2cm}p{3cm}p{2cm}rrrrr}\toprule
\textbf{Author} &\textbf{Modality \& slice thickness} &\textbf{Mean tumor size} &\textbf{Treatment} &\textbf{Target} &\textbf{\#Patients + Inclusion Criteria} &\textbf{Segmentation} &\textbf{Filtration} &\textbf{number of computed features} &\textbf{Retained features} &\textbf{Retained features category} &\textbf{Study endpoints} &\textbf{Results} &\textbf{\% RQS (points)} \\\midrule
Cozzi et al. &NECT 3 mm &- &Radiotherapy (volumetric modulated arc therapy) &* Prediction of (Local Control) LC \& OS for patients treated with arc therapy * lab assessment + CT or MRI imaging every 2 or 3 months the first 2 years, then every 6 months + RECISTs criteria &138 Patients with BCLC stages from A to C, Child-Pugh stages A-B with single lesion larger than 5 cm and multi-nodular lesions larger than 3cm were considered eligible for Radiotherapy &* Segmentation done using the CTV (clinical target volumes) which is manually contoured for the radiation treatment. * Whole tumor analysis (volume) &No filtration used &35 extracted features * 6 geometry and histogram based features (sphericity, compacity, skewness, kurtosis, entropy, energy) * 6 GLCM features * 2 NGLDM features * 11 GLRLM features * 10 GLZLM features &* Compacity (shape-based feature) correlated in association with BCLC stage with the OS * For the univariate analysis, they found energy (histo-based) and GLNU correlated with OS &Quantitative &OS \& local control of the tumor after radiation treatment &* AUC of the model is 0.80 * Survival could be predicted using a radiomics signature made by a single shape-based feature. &14 (5) \\
Zhou et al. &Contrast CT (30 and 60s) 1.25 mm &- &Hepatectomy &Recurrence &215 Patients Patients wo underwent partial hepatectomy were enrolled, with a follow up at leat within 1 year &Largest cross-sectional area of the tumor, manual delineation - exclusion of necrosis 2 experts &Filtration LoG, with 5 filter values [0, 1, 1.5, 2, 2.5] &300 features * Mean, SD, Kurtosis, Skewness; at 5 filters 5x4 features * Percent Mean & SD (10, 25, 50); at 5 filters 3x2x5 features * 5 features from 4 GLM at 5 filters 5x5x4 features Overall of 150 features per phases, and 2 phases -> 300 features &Radscore uses histogram features (skewness, energy, means...) &Quantitative &Recurrence &* Radiomics signature using first-order statistical features combined with clinical factors was a good predictor of early recurrence after surgery &25 (9) \\
Akai et al. &Contrast CT (27-28, 40 and 90 s) 5 mm &3.7 cm (2.4 - 7.0 cm) &Hepatectomy &DFS - OS &127 patients (initially 178 patients, but 16 without images, 31 with transcatheter arterial embolization excluded, 3 with too small lesions were excluded and 1 with insufficient timing on CECT were excluded) &Manually setting the region of interest to include the tumor within the slice at its max diameter. Single radiologist &Filtration LoG, with various SSF [0, 2, 4 and 6] &96 features * 6 features (mean, sd, positive calue pixels, entropy, kurtosis, skewness) * 4 SSF (0, 2, 4, 6) * 4 phases (unenhanced, AR, PV, delay) 6x4x4 = 96 features &The five most important features were entropy and histogram-based features (skewness and kurtosis) &Quantitative &OS and DFS &First-order statistical features were sufficient to predict postoperative survival &25 (9) \\
Chen et al. &Contrast CT (25 and 70s) 1.25 mm &- &Hepatectomy &OS and DFS &61 patients Patients should have only one lesion and their prospective survival above 3 months Baseline CT - 14 days before surgery &ROI was delineated around the tumor outline at the longest dimension (2 experts) &LoG with various sizes [0, 1, 1.5] &84 features * 12 Gabor features (from Scale 1.1 to 3.4) * 9 Wavelet features * 7 GLCM features at 3 filters (12+9+7) x 3 -> 84 features &* Textural features * Gabor and Wavelet responses were the key featurees associated with survival &Quantitative &OS and DFS &Tumor prognosis could be predicted using Gabor and Wavelet responses &17 (6) \\
Li et al. &Contrast CT (70s) 1.25 mm &8.0 cm (5.1 - 18.7 cm) &Hepatectomy or TACE &OS and treatment sensitivity &130 patients 86 treated by LR and 22 by TACE 104 with disease progression, 96 died by the end of the study &For each reader, a user-defined irregular ROI was drawn manually around the largest-cross sectional tumor outline. 2 radiologists &LoG with various sizes [0, 1, 1.5] &27 features Wavelet features: 3 levels, 3 directions, 3 LoG filters 3x3x3 = 27 features &* Only Wavelet features computed and 2 of them were related to survival &Quantitative &OS and Treatment sensitivity &Wavelet features were correlated with post operative survival suggesting a suitable treatment choice &19 (7) \\
Raman et al. &Contrast CT (25s) 3 mm &* Adenoma 7 ± 3cm * FNH 6 ± 3 cm * HCC 8 ± 3cm &- &Diagnosis &80 patients 17 FNH 19 Adenomas 25 HCCs 19 normal livers Patients must underwent dual-phase CT and having lesions of at least 2cm in the 3 directions &ROIs were selected from multiple axial slices to capture a range of representative textures from each lesion. The number of selected slices goes from 5 to 10. 2 observers &LoG with SSFs [0 2,3,4,5,6] &32 features * 5 features mean, SD, entropy, skewness, kurtosis * 6 filters 5x6 -> 30 features &Histogram based features mainly used to create the model, with key features being SD and Mean of histogram &Quantitative &Diagnosis &A model created using exclusively first-order statistical features was able to differenciate 3 types of hypervascular lesions. (They reached 15\% of error rate) &3 (1) \\
Kuo et al. &Contrast CT (30-35 and 60-70s) 2.5 mm &- &- &Treatment sensitivity and prediction of MVI &30 Patients * Images acquired within 1 month before surgical resection * no patients included received chemo before resection &No segmentation, Images analyzed visually by 2 experts (6 imaging features extracted by the experts) &No filtration used &6 imaging traits * Internal Arteries (binary) * Textural heterogeneity (AR phase) (0 to 4) * Wash-in max (0 - 4) * Wash-out max (0 - 4) * Necrosis (binary) * Tumor margin score (AR phase) (0 to 4) &No Computed features Only imaging traits selected a priori (Internal arteries, texture heterogeneity, wash-in, washout, necrosis present or absent, tumor margin score) were retained for the model creation Tumor margin showed a strong correlation with venous invasion and TNM stage while internal arteries showed correlation with venous invasion &Semantic &MVI status &The tumor margin showed strong correlation with MVI, TNM, and the expression of a drug response gene. While, internal arteries showed correlation with MVI &19 (7) \\
Banerjee et al. & \parbox{4cm}{Contrast CT (30-35, 60-70, 180-300 s) Thickness of 2.5-3 mm}&2.8 cm (1.8-4.5 cm) &Hepatectomy or LT &OS and RFS-based on RVI &157 patients Histologically confirmed HCC, treated by LR or LT and CECT within 12 months before surgery * 72 surgical resection * 85 liver transplantation MVI diagnosed in 45 patients &Only imaging features were evaluated to confirm the presence of RVI (radiogenomic venous invasion). 5 radiologists &No filtration &3 imaging traits * Internal arteries (binary) * Hypodense halo (binary) * Tumor-Liver difference (binary) &They found a correlation between RVI and MVI, and they computed RVI with imaging traits which were "Internal arteries", "hypodense halo", "tumor liver diffeence" &Semantic &OS and RFS &RVI (radiogenomic venous invasion) computed with three different imaging traits was correlated with MVI &53 (19) \\
Renzulli et al.&Contrast CT (25-30, 45-60, 180-300s) 2.5 mm &3.3 cm (1.8-5.2 cm) &Hepatectomy &Prediction of MVI based on TTPVI &125 patients Inclusion criteria: * preoperative imaging (CT or mRI) performed in their medical unit * HCC * hepatic resection was indicated Exclusion criteria : * patients with previous local-regional treatments * those who received treatment between imaging and hepatic resection &Only imaging features were evaluated by the experts. 2 radiologists &No filtration &5 imaging traits * Dimensions (below 2cm, between 2 and 5 cm, above 5cm) * Number of lesions (single, 2 or 3, more than 3) * Non-smooth margins * TTPVI (internal arteries + hypoattenuating halo) * Peritumoral enhancement &They only extracted imaging features from images and they found correlation between [tumor dimension, non-smooth margins, peritumoral enhancement and TPVI] and MVI for HCC patients &Semantic &MVI status &Tumor size, non-smooth tumor argins, peritumoral enhancement and TTPVI were correlated with the presence of MVI in HCC &8 (3) \\
Segal et al. &Contrast CT (threephasic) &- &Hepatectomy &OS and Prediction of MVI genes &\textbf{Training: 30 Test: 32} &They used only imaging traits (a total of 138) extracted by two radiologists. &No filtration &32 imaging traits selected among 138 common imaging traits of HCCs * Capsule * Wash-in, Wash-out, * Tumor-Liver difference ... &No computed features They found "internal arteries" as being a key imaging trait for predicting OS, and for building a two traits predictor of venous invasion in HCC in combination with "hypodense halo" &Semantic &OS and MVI &Internal arteries was found as being a key imaging feature for predicting OS and MVI in combination with hypodense halo. Those features were also correlated with the expression of genes involved in the development of HCC lesions &42 (15) \\
Zheng et al. &Contrast CT (22 and 60s) 5mm &- &Hepatectomy &Recurrence prediction and OS &Training: 212 Test: 107 Histologically confirmed HCC, preoperative CT, solitary HCC, patients without other malignancies nor anticancer therapy &ROI delinated around the tumor outline of the largest cross-sectinal area 2 radiologists &LoG at different sizes [1.0, 1.5, 2.0, 2.5] &110 GLCM related features * 22 features * 5 filters 22x5 -> 110 features &They only used textural features: GLCM features (see additional content for the exact 6 features) &Quantitative &Recurrence and OS &Radiomics score computed with textural features was sufficient to predict postoperative recurrence and survival in patients with solitary HCC &47 (17) \\
Peng et al. &Contrast CT (30, 60 and 120s) 5mm &Training: *MVI(+): 6.3 cm *MVI(-): 5.7 cm Validation: *MVI(+): 6.4 cm *MVI(-): 4.9 cm &- &Prediction of MVI &\textbf{Training: 184 Test: 120 (a) partial hepatectomy with tumor tissues pathologically confirmed to be HCC (b) validation of clinical data (c) validation of triphasic dynamic CT images acquired within 7 days prior to treatment (d) the presence of a single tumor.} &Images reviewed by 2 experts ROI semi-automatically segmented in the largest cross-sectional area → 980 features Imaging features for MVI status &No filtration &5 imaging traits * tumor margin * peritumoral enhancement * hypoattenuating halo * internal arteries * tumor-liver difference (binary) 490 rad features per phase (2 phases) -> 980 features * intensity direct; intensity histo; GLRLM; GLCM; neighbor intensity diff; shape &Radiological features : nonsmooth tumor margins, hypoattenuating halos and internal arteries + Eight radiomics features that were used to compute the radiomics signature (Entropy, shape, GLRLM, GLCM) &Semantic & Quantitative &MVI status &Radiological features and a radiomics signature computed with first-order statistical features showed correlation with MVI &47 (17) \\
Bakr et al. & \parbox{4cm}{Contrast CT (AR, PV, delay) optimal arterial opacification obtained using an automated bolus tracking technique Thickness $ \leq  $ 3mm }&7.4 cm &- &Prediction of MVI &28 patients * who underwent surgical resection of a previously untreated (locoregional therapy naive) presumed solitary HCC * Inclusion criteria included confirmed pathological diagnosis of HCC and availability of an adequate preoperative CT defined by a slice thickness $ \leq$ 3mm in all phases and obtained within 3 months of the surgery &3 ROIs were placed on different cross sections of the tumors, the first ROI was selected on a centrally located slice + 2 additional slices (one more superior and one more inferior but non contiguous) 4 radiologists &No filtration &464 features * 22 intensity related features * 333 Textural features * 7 shape related features * 102 Edge based features &They showed thay combining single-phase and delta-phase features can be correlated with the MVI compared to previous studies using semantic features They mainly rely on textural features but they used a lot of features to construct their model &Quantitative &MVI status &Textural features computed using single- or combined- phased images were correlated with MVI &3 (1) \\
Taouli et al. &Contrast CT (AR with bolus tracking, PV at 70s, Delay at 180s) &5.7 ± 3.2 cm &- &Prediction of MVI and aggressive phenotype &38 patients --> 26 CT/ 12 MRI * All patients underwent contrast-enhanced imaging, including CT (n = 26) or MR imaging (n = 12) * before liver resection (n = 36) or liver transplantation (n = 2). * The mean interval time between imaging and surgery was 19 days (range: 4–106 days) &Imaging traits + “slice-wise” evaluation for the enhancement ratio and the wash-out ratio. 2 radiologists &No filtration technique &11 imaging traits: * washin washout, hypovascularity... and 3 Quantitative features * enhancement ratios * Washout ratios * Tumor-to-liver contrast ratio &They found correlation between imaging traits such as [infiltrative pattern, mosaic appearance, presence of MVI, large size] and aggressive genotype &Semantic, Quantitative &signature of MVI and/or aggressive phenotype &Correlation was found between some imaging traits and the aggressive profile of the tumors &19 (7) \\
Xia et al. &Contrast CT (30, 55~70, 300 s) Thickness of 2.5~5mm &12 tumor below 5cm 26 above 5cm &Hepatectomy or LT &OS with interpretable biological meaning &38 patients (and 371 patients who had gene expression profiles) 47 patients from the 371 underwent CECT, but 9 were excluded because they received therapy with transarterial chemoembolization prior to imaging &Tumor was firstly delineated then divided into 3 spatially distinct sub-regions (using a multi-parametric clustering). 11 features were computed in each one of the 3 subregions and holistic features were computed in the whole tumor. Since the 3 spatially distinct subregions cover the entire tumor, we can consider this study as a whole tumor analysis. 1 radiologist &No filtration used &37 features * 4 first-order stat features * 2 geometry features * 5 textural features 3 subregions 3x11 --> 33 features And 4 features for the whole tumor 33+4 --> 37 features &2 features correlated with OS: "volume of transition region", and "cluster prominence" (only on the viable tumor), which compute the heterogeneity of the tumor &Quantitative &OS &The volume of transition between tumor and liver, and the heterogeneity of the lesion were correlated with survival. Those two features associated with six others were correlated with prognostic genes expression &22 (8) \\
\bottomrule
\end{tabular}
\end{table}
\end{landscape}

\renewcommand{\arraystretch}{5}
\normalsize