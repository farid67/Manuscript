\chapter{Conclusion}\label{conclusion}

In this thesis we successfully investigated deep learning based
approaches to better characterize hepatic tumors.

After describing the potential causes that can lead to the different
types of liver cancer, we motivated our choice to focus mainly on
hepatocellular carcinoma (HCC).

This primary cancer type, which is the most frequent one was described
through its origin, the way it is usually diagnosed, classified and
treated.

The different steps transforming healthy hepatic cells to cancerous ones
have been developed, along with the analysis of the physiological
changes often brought throughout the process.

To better characterize the liver tumors, we have decided to rely on
medical images solely, through a quantitative and objective analysis.

Regarding the modality choice, we retained the computed tomography
because it currently corresponds, with MRI, to the modality providing
the best quality of images for a non-invasive assessment of the disease.
CT was chosen over MRI mainly because it is usually the first choice for
less-specialized centers, and because it can be more easily interpreted
than MRI. With our medical studies review, we showed that the use of
dynamic temporal images is a prerequisite for imaging based liver cancer
research.

In the clinical practice, images are usually analyzed by the experts
with the naked eye, but the technological advancements allowed the
creation of computer assisted diagnosis tools (CADs), where a few number
of imaging features were initially used to differentiate benign and
malignant lesions.

In the present work, we compared the two paradigms allowing the computer
assisted analysis of medical images, either with engineered features, or
thanks to deep-learning. One of the key aspects of medical images
analysis being the segment, we compared the different semantic
segmentation architectures, and evaluated the benefit brought by
multiphase images which was most of the time neglected by previously
existing studies. Then we extracted relevant features from our semantic
segmentation architecture to tackle the prediction of the histological
grade. Each step of our research work required imaging databases
precisely annotated by experts in association with researchers to
improve their applicability.

A new technology called radiomics has been developed to compute a higher
number of features but it took a long time before this technology was
applied to the liver, especially because of the scarcity of publicly
available datasets.

We provided a detailed description of the radiomics pipeline, mainly
based on a manual segmentation of the ROI, followed by the extraction of
a high number of engineered quantitative features, thus being called
\emph{HCR} (Hand-Crafted Radiomics).

We presented our review, where a total of 15 \emph{HCR} studies
performed on HCC patients have been analyzed.

They were evaluated against the radiomics quality score (\emph{RQS})
which has been developed to assess the robustness and the
reproducibility of radiomics studies.

We pointed out the lack of reproducibility of the studies, with a mean
\emph{RQS} of 8.73 +/- 5.57 points out of a possible maximum value of 36
points.

Several important criteria were found as being ignored by the majority
of the studies, such as a prospective design, the use of open-sourced
data, the evaluation of the prediction on a validation dataset, or the
extraction of features at multiple timepoints.

The emergence of deep learning has chained the way a lot of imaging
related problems are comprehended.

The radiomics field has been also impacted by this novel set of
algorithms, and a new paradigm called \emph{DLR} (Deep-Learning
Radiomics) has been initiated, where one or several steps of the
radiomics pipeline are performed by deep learning algorithms.

We believe that the key part of the radiomics pipeline lies in the
segmentation of the ROI. This step has been found to suffer from a high
inter- and intra-observer variability, thus having consequences for the
accuracy of the final prediction.

We reviewed the different studies using deep learning architectures to
perform automatic segmentation of the liver and its tumors. We extracted
the common key settings shared by the majority of them, such as the
cascaded architecture or the use of a fully convolutional architecture.
However, we realized the lack of studies presenting results obtained
from multiphase images, which has been a key element in our work.

We performed the automatic segmentation of an internal dataset composed
of 104 sparse biphasic liver slices obtained from patients suffering
from HCC (images available before the injection of contrast medium:
Non-Enhanced CT, and at both arterial and portal venous phases).
Considering how challenging the segmentation of liver tissues is, the
amount of data, and the success of such an architecture, we decided to
train several specialized networks in a cascaded way.

When evaluating the performances of each specialized network, we
validated the hypothesis than the use of multiphase information allows a
better accuracy for each of the task than single phase based networks,
with significant difference obtained for the segmentation of the liver
(mean DSC of 89.9 +/- 15.6 for the multiphase network \emph{vs} 89.5 ±
13.2 for the best single phase network) and the active part of the
lesions (mean DSC of 75.5 ± 17.4 \emph{vs} 71.6 ± 20.7).

Regarding single phase networks, the \emph{PV} phase was the one
allowing the most accurate segmentation, with significant difference vs
\emph{AR} and \emph{NECT} for the segmentation of the parenchyma (mean
DSC of 88.7 ± 15.4) , the lesion (mean DSC of 87.8 ± 9.7), and both the
necrotic (mean DSC 77.8 ± 12.4) and the active (mean DSC of 71.6 ± 20.7)
part of the lesion.

We validated the hypothesis that several specialized networks combined
in a cascaded architecture perform better than a single network
addressing all the tasks simultaneously (obtained mean slice-wise DSC of
90.5 ± 13.2 for the parenchyma, 75.8 ± 15.1 for the necrosis and 59.6 ±
22.5 for the active part of the tumor when using the cascaded
architecture with the liver GT mask as input).

In a fully automatic manner (without using the liver GT mask), we were
able to reach promising results regarding the size of the dataset, with
a mean slice-wise DSC of 78.3 ± 22.1 for the parenchyma, 50.6 ± 24.6 for
the active tumor and 68.1 ± 23.2 for the necrotic part of the tumor.

We were also able to automatically compute the necrosis rate of the
tumors, with a mean error of 15.9\% when compared with the experts
obtained rates. This prediction is accurate enough to consider the
obtained necrosis rate when evaluating the treatment outcomes.

To overcome the limited size of multiphase datasets, we define a
strategy to ``augment'' the weakly annotated or non-annotated available
datasets.

With a robust registration algorithm and our cascaded architecture where
specialized networks are trained on a sufficient amount of data, we were
able to perform the semantic segmentation of liver and its tumors on
unseen cases.

We validated this assumption by performing the segmentation of TCIA
tumors, and obtained a mean patient-wise DSC of 73.2 ± 20.6.

To predict the histological grade, we proved that features learned by
our mutliphase semantic segmentation network are relevant to perform
this task.

We predicted the grade for central tumor slices, and after a CV
training, we correctly predicted 74\% of them. When considering the
patient histological grade as being the most frequent one in the central
tumor slices, we were able to correctly classify 15 patients over the 18
of the dataset.

As a conclusion, our research work is the first to propose a fully
automatic \emph{DLR} pipeline with multiphase images as input. Our
strategy is dedicated to small databases where the entire annotations
are often not available but the same workflow can be applied for larger
datasets. Our \emph{DLR} architecture was dedicated, in this research
work, to the prediction of the histological grade of HCC patients, but
our strategy can be extended to other characteristics of HCC or other
types of liver cancers.

Regarding the lack of publicly available datasets, we encourage the
creation of an open 3D CECT dataset containing liver volumes of both
healthy and diseased patients with precise and complete pathological
information, and expert delineations of liver, hepatic vessels and
lesions for each phase.
